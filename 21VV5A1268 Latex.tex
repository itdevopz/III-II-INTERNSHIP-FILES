% Generated by GrindEQ Word-to-LaTeX 
\documentclass{article} % use \documentstyle for old LaTeX compilers

\usepackage[utf8]{inputenc} % 'cp1252'-Western, 'cp1251'-Cyrillic, etc.
\usepackage[english]{babel} % 'french', 'german', 'spanish', 'danish', etc.
\usepackage{amsmath}
\usepackage{amssymb}
\usepackage{txfonts}
\usepackage{mathdots}
\usepackage[classicReIm]{kpfonts}
\usepackage{graphicx}

% You can include more LaTeX packages here 


\begin{document}

%\selectlanguage{english} % remove comment delimiter ('%') and select language if required


\noindent 

\noindent 

\noindent 

\noindent 

\noindent \includegraphics*[width=7.13in, height=11.74in]{image1}\eject 

\begin{tabular}{|p{4.2in}|} \hline 
\newline \newline \newline \newline \newline \newline \textbf{PROGRAM BOOK FOR\newline SEMESTER INTERNSHIP\newline \newline \newline }\newline Name of the Student: DANNANA SAI AJITH KUMAR\newline \newline Name of the College: JNTU Gurajada Vizianagaram\newline College of Engineering(A)\newline \newline Registration Number:  21VV5A1268\newline \newline Period of Internship:   From:08-May-2023\newline To: 01-July-2023\newline \newline Name \& Address of the Intern Organization:\newline Rasthriya Ispat Nigam Limited, Visakhapatnam Steel Plant,\newline Visakhapatnam\newline \newline \newline \newline \newline \newline JAWAHARLAL NEHRU TECHNOLOGICAL UNIVERSITY\newline III Year\newline  \\ \hline 
\end{tabular}



\begin{tabular}{|p{4.2in}|} \hline 
\newline \newline \newline \textbf{An Internship Report on\newline }\newline \textbf{DATA ANALYTICS AND FST WEB \newline DEVELOPMENT USING PYTHON\newline }\newline Submitted in accordance with the requirement for the degree of\newline III B.Tech\newline \newline Under the Faculty Guideship of\newline \textbf{T.V. Kameswara Rao \newline }\newline Department of\newline \textbf{INFORMATION TECHNOLOGY\newline JNTU Gurajada Vizianagaram \newline College of Engineering(A)\newline }\newline Submitted by:\newline \newline DANNANA SAI AJITH KUMAR\newline Reg. No: 21VV5A1268\newline Department of IFORMATION TECHNOLOGY\newline JNTU Gurajada Vizianagaram \newline College of Engineering(A)\newline  \\ \hline 
\end{tabular}



\noindent 

\noindent 

\noindent Student`s Declaration

\noindent 

\noindent I, DANNANA SAI AJITH KUMAR a student of B.TECH Program, Reg. No. 21VV5A1268 of the Department of INFORMATION TECHNOLOGY JNTU GURAJADA VIZIANAGARAM CEV(A) College do hereby declare that I have completed the mandatory internship from 08-MAY-2023 to 01-JULY-2023 in RASHTRIYA ISPAT NIGAM LIMITED VIZAG STEEL PLANT under the Faculty Guideship of M GEETHA MADHURI Department of INFORMATION TECHNOLOGY, JNTU GURAJADA VIZIANAGARAM CEV(A)

\noindent 

\noindent (Signature and Date)

\noindent 

\noindent \eject 

\noindent 

\noindent 

\noindent Acknowledgements

\noindent \textbf{}

This acknowledgement transcends the reality of formality when I express deep gratitude and respect to all those people behind the screen who inspired and helped us in completion of this project work.

I take the privilege to express my heartfelt gratitude to my guide \textbf{Ms. M Geetha Madhuri, }Assistant Professor(c), Department of Information Technology, JNTUG- Vizianagaram for his valuable suggestions and constant motivation that greatly helped me in successful completion of the dissertation. Wholehearted cooperation and the keen interest shown by him at all stages are beyond words of gratitude.

With great pleasure and privilege, I wish to express my heartfelt sense of gratitude and indebtedness to \textbf{Dr. B. TIRIMULA RAO}, Assistant Professor \& Head of Department of Information Technology, JNTUG --Vizianagaram, for his supervision.

I extend heartfelt thanks to our principal Prof. \textbf{Dr. K. SRIKUMAR }for providing intensive support throughout my dissertation.

I am also thankful to all the Teaching and Non-Teaching staff of Information Technology Department, JNTUG - Vizianagaram, for their direct and indirect help provided to me in completing the dissertation.

I extend my thanks to my parents and friends for their help and encouragement for the success of my dissertation

\noindent 

\noindent \textbf{DANNANA SAI AJITH KUMAR}

      (21VV5A1268)

\noindent 

\textbf{Official Certification}

\noindent \eject 

\noindent 

\textbf{Certificate from Intern Organization }

\noindent \includegraphics*[width=5.73in, height=7.77in]{image2}

\noindent \eject 

\noindent 

\noindent Contents

\begin{tabular}{|p{0.5in}|p{3.1in}|p{0.8in}|} \hline 
s.no. & Contents & page no \\ \hline 
1. & Executive Summary &  \\ \hline 
2. & Overview of the RiNL VSP &  \\ \hline 
 & 2.1 About RINL &  \\ \hline 
 & 2.2 About L\&DC PTMS &  \\ \hline 
3. & Internship &  \\ \hline 
4. & Activity log and weekly reports &  \\ \hline 
 & 4.1 Activity log for first week &  \\ \hline 
 & 4.2 Activity log for second week &  \\ \hline 
 & 4.3 Activity log for third week &  \\ \hline 
 & 4.4 Activity log for forth week &  \\ \hline 
 & 4.5 Activity log for fifth week &  \\ \hline 
 & 4.6 Activity log for sixth week &  \\ \hline 
 & 4.7 Activity log for seventh week &  \\ \hline 
 & 4.8 Activity log for eighth week &  \\ \hline 
5. & Outcomes description &  \\ \hline 
 & 5.1 Describe the work environment you have experienced &  \\ \hline 
 & 5.2 Describing the real time technical skills that were acquired &  \\ \hline 
 & 5.3 Describe how you could improve your communication skills. &  \\ \hline 
\end{tabular}



\begin{tabular}{|p{0.7in}|p{2.8in}|p{0.8in}|} \hline 
 & 5.4 Describe how could you could enhance your abilities in group discussions, participation in teams, contribution as a team member, leading a team/activity. &  \\ \hline 
 & 5.5 Describe the technological developments you have observed and relevant to the subject area of training. &  \\ \hline 
 & 5.6 Describe the technological developments you have observed and relevant to the subject area of training. &  \\ \hline 
\end{tabular}



\noindent 

\noindent \eject 

\noindent 

\noindent CHAPTER 1: EXECUTIVE SUMMARY

\noindent 

\noindent This abstract provides a summary of a comprehensive analysis conducted on the materials consumed in steel plant over a three-month period. The study aimed to evaluate the usage patterns, identify potential areas for optimization in the plant. The analysis utilized historical data from the company's records, including procurement and consumption information of various materials used in the steel production process. 

\noindent This database consists of the raw materials like iron ore, coal, and limestone, as well as consumables like refractory's, lubricants, and chemicals. By analysing the consumption rates and change in consumption of materials in these three months, the study identified materials that were utilized at a higher or lower rate than anticipated, highlighting potential areas for improvement.

\noindent  The analysis also explored the variance between opening value and closing value, to assess the increase and decrease in stock inventory. By examining the data, trends and patterns emerged, offering opportunities to optimize material management practices and reduce waste. 

\noindent The outcomes of this analysis have significant implications for the steel plant's operations and cost management. The findings can serve as a basis for informed decision-making, aiding in strategic planning, procurement optimization, and inventory management. In conclusion, this analysis provides valuable insights into the material consumption patterns within the steel plant, emphasizing areas for improvement and optimization. The findings have the potential to reduce costs, and enhance the overall sustainability of the company's operations.

\noindent \eject 

\noindent 

\noindent CHAPTER 2: Overview of RINL VSP

\noindent 2.1 About RINL

\noindent Rashtriya Ispat Nigam Ltd, (abbreviated as RINL), also known as Vizag Steel Plant, is a public Steel producer based in Visakhapatnam, India. Rashtriya Ispat Nigam Limited (RINL) is the corporate entity of Visakhapatnam Steel Plant (VSP), India's first shore-based integrated steel plant built with state-of-the-art technology. It is founded in 1971; the plant focuses on producing value-added steel, producing 5.773 million tonnes of hot metal, 5.272 million tonnes of crude steel and 5.138 million tonnes of saleable steel. 

\noindent Visakhapatnam Steel Plant (VSP) is a 7.3 MTPA plant. It was commissioned in 1992 with a capacity of 3.0 MTPA of liquid steel. The company subsequently completed its capacity expansion to 6.3 MTPA in April 2015 and to 7.3 MTPA in December 2017. 

\noindent The company is having one subsidiary, viz. Eastern Investment Limited (EIL) with 51\% shareholding, which in turn is having two subsidiaries, viz. M/S Orissa Mineral Development Company Ltd (OMDC) and ws Bisra Stone Lime Company Ltd (BSLC). 

\noindent The company has a partnership in RINMOIL Ferro Alloys Private Limited and International Coal Ventures Limited in the form of Joint Ventures with 50\% and 26.49\% shareholding respectively. 

\noindent The decision of the Government of India to set up an integrated steel plant at Visakhapatnam was announced by then Prime Minister Smt. Indira Gandhi in Parliament on 17 January 1971. 

\noindent VSP is the first coastal-based integrated steel plant in India, 16km west of the city of destiny, Vishakhapatnam, bestowed with modem technologies; VSP has an installed capacity of 3 million tons per annum of liquid steel and 2.656 million tons of saleable steel. 

\noindent The saleable steel here is in the form of wire rod coils, Structural, Special Steel, Rebar, Forged Rounds, etc. At VSP, there lies emphasis on total automation, seamless integration and efficient up gradation. This result in a wide range of long and structural products to meet stringent demands of disceming customers in India \& abroad; SP product meets exalting international Quality Standards such as IIS, DIN, BIS, BS, etc.RINL---VSP was awarded 'Star Trading HOUSE" status during 1997-2000 having established a fairly dependable export market, VSP Plans to make a continuous presence in the export market.

\noindent 

\noindent Different sections at the RINL VSP: 

\begin{enumerate}
\item  Coke oven and coal chemicals plant 

\item  Sinter plant 

\item  Blast Furnace 

\item  Steel Melt Shop 

\item  Continuous casting machine 

\item  Light and medium machine mills 

\item  Calcimine and refractive materials plant 

\item  Rolling mills 

\item  Thermal power plant 

\item  Chemical power plant
\end{enumerate}

\noindent 

\noindent 2.2 About L\&DC PTMS

\noindent L\&DC:

\noindent A Learning and Development Center is a facility or department within an organization that is responsible for designing, implementing, and managing training and development programs for employees. The centre's primary goal is to enhance the skills, knowledge, and competencies of employees, which can lead to improved job performance, increased productivity, and career growth.

\noindent Learning and Development Centre's offer various training initiatives, such as workshops, seminars, e-learning modules, on-the-job training, leadership development programs, and other educational opportunities. These programs are often tailored to the specific needs of the organization and its employees, focusing on areas such as technical skills, soft skills, compliance training, and professional development.

\noindent 

\noindent PTMS:

\noindent Project Trainee Management System(PTMS) of Rashtriya Ispat Nigam Limited - Visakhapatnam Steel Plant. This online system aims to ease the whole process of getting the Industrial Project Based Training/Internship at the organization.

\noindent It is medium for applying and getting opportunity to do training or internship at RINL VSP

\noindent 

\noindent \eject 

\noindent 

\noindent CHAPTER 3: INternship 

\noindent 

\noindent This project aims to utilize data analytics techniques to gain deeper insights into the patterns, trends, and factors influencing material consumption within our steel plant. By analysing vast amounts of data collected from various sources, such as production logs, inventory records, and supply chain information, we seek to identify opportunities for optimization, cost reduction, and sustainable resource management. 

\noindent The integration of data analytics and enterprise resource planning (ERP) systems will be a key focus of this project. The ERP system serves as a centralized repository of information, encompassing all aspects of the steel plant's operations. By integrating data analytics capabilities into the existing ERP system, we aim to enable real-time monitoring, predictive analytics, and informed decision-making across departments, ensuring the efficient utilization of materials throughout the plant.

\noindent The objectives of this project include: 

\begin{enumerate}
\item  Analysing historical material consumption data to identify patterns and correlations. 

\item  Developing predictive models to forecast material requirements based on production schedules, market demands, and other relevant factors.

\item  Implementing real-time monitoring and alerts to detect anomalies and deviations in material consumption, facilitating proactive intervention. 

\item   Recommending strategies for optimizing material usage, reducing waste, and minimizing costs without compromising product quality. 
\end{enumerate}

\noindent The successful implementation of this project will result in significant benefits for our steel plant. These include improved resource planning, enhanced operational efficiency, reduced material waste, and ultimately, a positive impact on our bottom line. We extend our appreciation to the IT and ERP department, who will be leading this project, as well as to the management and stakeholders for their support and investment in data analytics capabilities. 

\noindent We also acknowledge the invaluable contributions of data analysts, IT professionals, and subject matter experts who will collaborate on this project. By leveraging the power of data analytics in material consumption, our steel plant aims to remain at the forefront of innovation, competitiveness, and sustainability in the steel industry.


\end{document}

